\chapter{Spatial and Mobility Data}

The study of human mobility data has a wide range of applications, such as urban planning, epidemic control, transportation management, immigration monitoring. The last decade has seen a rapid increase of digital traces that represent human movements. Examples of this data are those generated by GPS devices embedded in phones or cars, mobile phone records collected by telco companies, and geotagged social media posts. This chapter will describe the main characteristics of such data.

\section{GPS Data}

\textbf{Global Navigation Satellite Systems} (\textbf{GNSS}) use satellites to provide geo-spatial positioning of objects on the Earth's surface in terms of longitude, latitude, and altitude. There are currently multiple systems managed by different countries of the world. The most famous GNSS is the US \textbf{Global Positioning System} (\textbf{GPS}), and receivers are now embedded in many commonly used tools such as smartphones, cars, and wearable devices. 

GPS can be thought of a system of three main components: a space segment, a control segment, and a user segment. The \textbf{space segment} consist of a constellation of 24+ satellites orbiting around the planet, circling it twice a day in six equally spaced orbits. Each orbit has four satellites, and any user will always see at least four satellites from any point on Earth. The \textbf{control segment} consist of all the ground facilities used to track/command the satellites and monitor their transmissions. The \textbf{user segment} is represented by the GPS receivers in the user devices. On mobile phones, GPS receivers are activated by software that requires the user's position. On cars, receivers automatically turn on as the car starts. Signals are sent to a server every few seconds up to every few minutes; their precision can range from a few centimeters to meters, depending on the quality of the device.

To identify a point on Earth, a total of four satellites are used (three to identify the point, a fourth one for redundancy): each satellite broadcasts a radio signal at the speed of light that contains information about its location, status, and time of the on-board atomic clock. When the signal hits the device receiver, the time of arrival is compared to that of the send to calculate the distance $d$ from the satellite.

Using this distance, we can imagine a sphere of radius $d$ centered around the satellite. The intersection of all the spheres centered around the involved satellites is the exact location of the device. 
\\
The GPS traces format is as follows:
\begin{center}
$(u, lat, lng, alt, t)$
\end{center}
where:
\begin{itemize}[itemsep=-5pt, label=-]
    \item \textit{u} is the user/receiver ID;
    \item \textit{lat} is the latitude;
    \item \textit{lng} is the longitude;
    \item \textit{alt} is the altitude;
    \item \textit{t} is the timestamp.
\end{itemize}

\paragraph{Pros}
Ubiquitous, produces high resolution, dense traces.

\paragraph{Cons}
Rarely publicly available, no semantic information (e.g., what the user is doing, if he/she is in a building or outside), prone to errors when signal is noisy or absent, and small samples.

\section{Mobile Phone Records}

Mobile phone coverage reaches almost 100\% of the population of most countries. Telco companies collect the activity of phone users for billing purposes, storing information about where users are located, when they communicate with others, and who they communicate with. This type of information can be split in the following three groups.

\paragraph{Call Detail Records} CDR are generated each time a user makes/receives a call or an SMS. A CDR is a tuple
\begin{center}
$(n_o, n_i, t, A_o, A_i, d)$
\end{center}
where:
\begin{itemize}[itemsep=-5pt, label=-]
    \item $n_o$ is the caller's ID (can be the phone number, but it should be anonymized for privacy reasons);
    \item $n_i$ is the receiver's ID (same as above);
    \item $t$ is the timestamo if the call;
    \item $A_o$ is the ID of the antenna the caller is connected to;
    \item $A_i$ is the ID of the antenna the receiver is connected to;
    \item $d$ is the duration of the call.
\end{itemize}

\paragraph{eXtended Detail Records}
XDR are generated when a device uploads or downloads data from the Internet. These connections can be both human-triggered and device-triggered (e.g., software updates, background synchronization). An XDR is a tuple
\begin{center}
$(n, t, A, k)$
\end{center}
where:
\begin{itemize}[itemsep=-5pt, label=-]
    \item $n$ is the user's ID;
    \item $t$ is the timestamp of the connection;
    \item $A$ is the ID of the antenna the user is connected to;
    \item $k$ is the amount of KB of data transferred.
\end{itemize}

\paragraph{Control Plane Records}
CPR are used to check the status of the network, and are generated each time a device connects to a new antenna. These records are exclusively network-triggered. A CPR is a tuple
\begin{center}
    $(n, t, A, e_1, e_2, \dots, e_n)$
\end{center}
where
\begin{itemize}[itemsep=-5pt, label=-]
    \item $n$ is the user's ID;
    \item $t$ is the timestamp of the connection;
    \item $A$ is the ID of the antenna the user is connected to;
    \item $e_i$ are the ``events'' that happen in the network.
\end{itemize}

\paragraph{Pros}
Ubiquitous, huge sample size, rich and multidimensional (give information about space, time, and social connections among users, since we can know who calls/messages who).

\paragraph{Cons}
Not publicly available, much more sparse and low resolution than GPS, suffers from ping-pong effect (since areas of influence of antennas are not perfectly regular and may overlap, a device may rapidly flip between them).

\begin{figure}[!ht]
    \centering
    \includesvg[width=\textwidth]{img/mobile_data_comparison.svg}
    \caption{Comparison of densities of different data types.}
    \label{fig:mobile-data-comparison}
\end{figure}

\section{Location-based Social Networks}

Certain social media platforms allow users to geotag their posts, i.e., assign geographic and temporal information. This information can be represented by absolute data (latitude, longitude), relative data (distance from some point of interest), or symbolic (home, office, stadium, shopping mall).
This geospatial information can then be easily coupled with the social network structures of connected users, allowing us to consider the social dimension in our analysis. A LBSN record is a tuple
\begin{center}
    $(u, t, lid, lpos, post)$
\end{center}
where:
\begin{itemize}[itemsep=-5pt, label=-]
    \item $u$ is the user's ID;
    \item $t$ is the timestamp of the post;
    \item $lid$ is the location ID;
    \item $lpos$ is the location position;
    \item $post$ is the content of the text post.
\end{itemize}

\paragraph{Pros}
Ubiquitous, can provide precise information with little error, carries semantic information.

\paragraph{Cons}
Is very sparse (even more than mobile phone records), suffers from self-selection bias (users that frequently produce such content tend to belong to a specific demographic).

\section{Road Networks and Points of Interest}

\textbf{Road networks} describe streets and their intersections. They are usually represented by multigraphs, where a node is an intersection and an edge is a road. Nodes and edges are geometries, encoded as Point, Lines, or Polygons. A road network is described by a two sets of records: the first contains all the nodes, the second all the edges. A node record is a tuple
\begin{center}
    $(id, lat, lng, streets, geom)$
\end{center}
where:
\begin{itemize}[itemsep=-5pt, label=-]
    \item $id$ is the node ID;
    \item $lat, lng$ are the geographic coordinates;
    \item $streets$ number of streets crossing in that intersection;
    \item $geom$ is the geometric shape of the node (usually a Point).
\end{itemize}
An edge record is a tuple
\begin{center}
    $(id, u, v, f_1, \dots, f_k, geom)$
\end{center}
where:
\begin{itemize}[itemsep=-5pt, label=-]
    \item $id$ is the edge ID;
    \item $u$ is the origin node;
    \item $v$ is the destination node;
    \item $f_i, \dots, f_k$ are the road characteristics (e.g., if the road is one-way only, its length, number of lanes, name, speed limit);
    \item $geom$ is the geometric shape of the edge (usually a Line).
\end{itemize}
Road networks help associate GSP traces to actual streets, and can find routes between two places. These structuresa are also aften used in simulations for traffic management and urban planning.

\textbf{Points of interest} describe certain geospatial entities in a city, such as grocery stores, transit stops, restaurants, etc. They carry semantic information about activities in cities and associated human behaviour. A POI record is a tuple
\begin{center}
    $(id, f_1, \dots, f_n, geom)$
\end{center}
where:
\begin{itemize}[itemsep=-5pt, label=-]
    \item $id$ is the POI ID;
    \item $f_i, \dots, f_n$ are the POI characteristics (e.g., name, type of building, road nodes involved);
    \item $geom$ is the geometric shape of the POI (usually a Point or a Polygon).
\end{itemize}